\capitulo{2}{Objetivos del proyecto}

A continuación se pasará a detallar los objetivos que han motivado la realización de este proyecto, tanto a nivel global como técnico y personal.

\section{Objetivos globales}

\begin{itemize}
	\item Automatizar el proceso de extracción de los datos bibliográficos de las bases de publicaciones anteriormente mencionadas.
	
	\item Dotar a la información bibliográfica extraída del formato propio de estos documentos (BibTex, RIS,EndNote).
	
	\item Automatizar el proceso de subida de los datos extraídos a la aplicación ACADEMIA.
	
	\item Desarrollar una aplicación de Escritorio que permita al usuario una fácil interacción con los procesos automatizados.
	
	\item Almacenar la información Extraída para un posible uso futuro.
\end{itemize}

\section{Objetivos técnicos}

\begin{itemize}
	\item Desarrollar una aplicación utilizando Python para la extracción de datos, así como la subida de estos a la aplicación.
	
	\item Utilizar Scrum como metodología de planificación del proyecto.
	
	\item Utilizar Git como sistema de control de versiones junto con la plataforma GitHub.
	\item Utilizar LaTeX como herramienta de documentación.
	
	\item Realizar test unitarios.
\end{itemize}

\section{Objetivos personales}

\begin{itemize}

	\item Realizar una aportación al sector de la investigación en la universidad.
	
	 \item Adquirir conocimientos en nuevas materias no tratadas durante la carrera.
	 
	 \item Explorar nuevos conceptos de trabajo y desarrollo.
	 
\end{itemize}
