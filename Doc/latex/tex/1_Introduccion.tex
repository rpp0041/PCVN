\capitulo{1}{Introducción}

El trabajo de investigación y desarrollo es una de las principales labores de los miembros de la universidad, un trabajo que permite la adquisición de nuevos conocimientos y teorías sobre el entorno que nos rodea mediante la interacción directa del sujeto con este. Estos conocimientos son muy valiosos pues no pueden ser adquiridos de ninguna otra forma, es por esto por lo que se debe apoyar la investigación en los ámbitos universitarios.

Este trabajo requiere de tiempo y esfuerzo para ser llevado a cabo y debidamente redactado y plasmado. Para después ser publicado en las revistas de divulgación científica, universidades etc. Tratando así de dar a conocer los nuevos hallazgos o conclusiones extraídas de la investigación. Es frecuente que los investigadores deseen acceder a los cuerpos docentes universitarios, pues se entiende que desean compartir los conocimientos adquiridos a través del tiempo y la investigación, para ello deben pasar por un proceso de evaluación y acreditación por la Agencia Nacional de Evaluación y Acreditación (ANECA). Este proceso implica aún más tiempo y esfuerzo, en la labor de reunir y tratar toda la información referente a las publicaciones realizadas por un autor, para posteriormente subir toda esa información al programa que para ello tiene habilitado la ANECA (ACADEMIA).

ACADEMIA lleva a cabo el proceso de evaluación curricular para la obtención de la acreditación para el acceso a los cuerpos docentes universitarios de Profesor Titular de Universidad y Catedrático de Universidad.
Incluye el procedimiento para la exención del requisito de pertenecer al Cuerpo de Profesores Titulares de Universidad a que se refiere el art. 60.1 de la Ley Orgánica 6/2001, de 21 de diciembre. [Cita a bibliografía]. El procedimiento de acreditación tiene abierta la presentación de solicitudes a través de la Sede Electrónica del MECD [Cita a bibliografía], mediante el uso de una aplicación informática [Cita], la cual permite esencialmente la presentación de solicitudes y la cumplimentación del CV.
En esto último es en lo que se enfoca principalmente este proyecto, en ayudar a los investigadores a reunir adecuadamente los datos bibliográficos referentes a las publicaciones realizadas para posteriormente realizar una subida de estos datos a la propia aplicación de la ANECA , con el correspondiente formato. Evitando así este tedioso proceso al usuario.

Así en este proyecto podemos diferenciar dos funciones 
	
	Recopilar: Para recopilar la máxima cantidad de información posible sobre las publicaciones de un autor, consultaremos las principales bases de datos del ámbito de la investigación.
		Google Scholar
		Scopus
		Web of Science

	Enviar: Esta función no se compondrá únicamente de enviar la información a la aplicación de la ANECA, si no también de agrupar toda la información obtenida , dotarla de una estructura y formato adecuado y finalmente enviarla a esta aplicación.
	

