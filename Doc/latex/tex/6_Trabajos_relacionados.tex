\capitulo{6}{Trabajos relacionados}

En cuanto al apartado de la extracción de datos provenientes de las páginas ya mencionadas, lo más parecido que se podría encontrar serían las propias librerías ya usadas para este proyecto, así como algunas otras de funcionalidad muy similar

\section{Google Scholar}
scholar.py [1] Módulo de \emph{Python} que permite realizar peticiones y analiza las respuestas para devolver únicamente la información relevante
\section{Scopus}
scopus-api [2] Módulo de \emph{Python} que interactúa con la propia \emph{API} de \emph{Scopus}	para analizar los datos extraídos de las peticiones y devolver los datos relevantes con una estructura \emph{pandas DataFrame}[3]
\section{Web of Science}
isi [4] Herramientas que combinan \emph{Python} y scripts en \emph{bash} para acceder a \emph{Web of science} y extraer la información.

En cuanto al proceso de subida de los datos a la aplicación \emph{ACADEMIA}, tras una breve búsqueda por Internet , podemos encontrar que no existe ninguna aplicación o script que realice esta labor, al menos de manera pública. Puede darse el caso de que algún investigador haya desarrollado algún tipo de script para realizar una labor similar al propósito de este proyecto, pero con carácter personal y privado.

Como vemos no existe nada que se parezca en su totalidad a la idea de este proyecto, que combine las 2 ideas fundamentales de extraer y enviar	