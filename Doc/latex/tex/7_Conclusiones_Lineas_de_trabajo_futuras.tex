\capitulo{7}{Conclusiones y Líneas de trabajo futuras}

En este apartado vamos a exponer las conclusiones extraídas de la realización de este proyecto y las posibles líneas futuras de desarrollo para la continuidad del proyecto.

\section{Conclusiones}

Una vez finalizado el proyecto podemos decir que:
\begin{itemize}
	\item El objetivo en líneas generales del proyecto se ha cumplido satisfactoriamente, se ha conseguido crear una herramienta funcional que permita a los investigadores recolectar y actualizar la información referente a su curricular de una manera sencilla e intuitiva.
	\item Ha sido satisfactorio comprobar que las técnicas y conocimientos aprendidos a lo largo de la carrera han sido útiles. 
	\item Por otra parte, otro de los objetivos de este proyecto era adquirir nuevos conocimientos y técnicas, objetivo que considero satisfactoriamente cumplidos. Se ha profundizado en las técnicas de \emph{Web Scraping}, tratado de datos bibliográficos y la planificación y documentación de proyectos.
	\item Gracias a los distintos problemas encontrados durante el desarrollo, se ha podido aprender acerca del tratamiento de los imprevistos y los problemas, así como en la búsqueda de soluciones y alternativas. Conceptos, aunque quizás no tan relacionados con el grado en sí, se entiende que son valiosos y necesarios para el desarrollo de una carrera profesional.
\end{itemize}

\section{Líneas de trabajo futuras}
Cabe aclarar que, aunque se proceda a la entrega del Trabajo Fin de Grado (TFG), esto no quiere decir que sea un proyecto totalmente perfecto y cerrado, sino que hay ciertos aspectos que se pueden mejorar para mejorar la funcionalidad y la experiencia del usuario:
\begin{itemize}
	\item Dotar a la aplicación de una interfaz renovada, mas \emph{"moderna"} mediante el uso de un fichero CSS que vaya más allá de las limitaciones que tiene la librería usada (Tkinter), la cual proporciona herramientas suficientes para ser funcional, pero queda algo anticuada con respecto a otras aplicaciones de uso cotidiano.
	\item Añadir una funcionalidad que permita, no solo subir la información bibliográfica sino también el propio texto de la aplicación, como forma de aportar más datos e información para el proceso de acreditación y evaluación.
	\item Mejorar el número de datos que tiene actualmente el objeto de \emph{Python} que contiene los datos referentes al índice de impacto de las distintas revistas, aumentando en todo lo posible el número de revistas de las que posee la información para que el número de consultas a la \emph{JCR InCites} sea el mínimo posible , haciendo que el proceso más ágil.
\end{itemize} 